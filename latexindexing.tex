\documentclass{amsart}
\usepackage{amsmath}
\usepackage{amssymb}
\usepackage{tikz}

\title{Dynamic Indexing of \LaTeX\ documents}
\author{L.A.H.}

\newcommand{\idx}{\iota}

\begin{document}

\maketitle

In this article, I will discuss an application of indexing technique to index interesting position in \LaTeX\ document, in particular: starting and ending points of
\begin{enumerate}
\item Comments
\item Special symbols such as curly braces
\item Command (i.e. \emph{control sequences} in \TeX\ terminologies) and some subclass of commands such as sectioning, environments or file inclusion
\item Starting and ending of formulas
\end{enumerate}

For simplicity, let us ignore complicated commands such as \texttt{catcode} or verbatim environment in the following discussion.

At first sight, indexing comments and special symbols seems to be easiest of the above since \LaTeX\ comment starts with simply a percentage (\texttt{\%}) character and spans all the way to the end of the line. The problem is that: a special symbols can be \emph{escaped with backslash} (\texttt{\textbackslash}) character and \emph{backslash itself can also be backslash-escaped} as well. Beware that there is a circular dependency here: backslash only functions as escaped characters when it is \emph{not} under line comment block!

In general, one has a rule: if a character is not in comment block and the maximum number of consecutive backslashes right before a character is even then the character is not backslash-escaped.

A solution (to the first three) is to use multiple multilevel indexers such as
$$\idx_p, \idx_l, \idx_b, \idx_s$$
which locates all occurrences of \%, linefeed (EOL), potentially\footnote{Meaning: those in comment blocks as well as in normal code.} non-escaped backslashes and other special symbols of interest. Now I have mentioned that index are also  sequences of (increasing) numbers and thus, can be further indexed. We build the comment index for the original sequence by indexing the percentage-index $\idx_p$. Compared to indexing the comment blocks directly from the text, this is much simpler: the decision only depends on the previous character (whether it is a backslash or not) and whether it is escaped (efficiently checked using the $\idx_b$ indexer). The ending of comment block is just the first EOL after the percentage, efficiently obtained with $\idx_l$.

Actual special symbols and control sequences can be likewise indexed by indexing $\idx_b$ and $\idx_s$: the decision is now point-wise: we just have to confirm that the symbol is NOT comment block. Note that when $\idx_b$ is modified, one has to modify its index!

At first sight, the first two dynamic indexers and the last one are easily obtained since the indexing predicate depends only on the position while the third one $\idx_{b}$ is much more complicated since its indexing predicate, say $\alpha_{b}$, depends on a backward range of undetermined length. Here is the catch: $\alpha_{b}(x)$ actually depends on the previous character and $\alpha_{b}(x-1)$. In particular, if the previous character i.e. character at $x-1$ is not a backslash, evidently $\alpha_{b}(x)$ should be true. Otherwise (previous character is a backslash) then $\alpha_{b}(x)$ is the negation of $\alpha_{b}(x - 1)$ which can be checked efficiently if the index data structure admits \emph{efficient membership checking}. Utilizing this view point, the (pseudo) decision region at any point $x$ is just the two points in $[x-1,x+1)$ so when a subsequence $[s..e)$ is replaced, we only need to revalidate the (adjusted) range $[s..e')$ where $e'$ is the first non-backslash after $e$.

\end{document}